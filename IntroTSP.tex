\section{What is the Travelling Salesman Problem?}
\label{IntroTSP}

\par
In 1832, a German handbook for travelling salesmen was published, titled ''Der Handlungsreisende - wie er sein soll und was er zu tun hat, um Auftr{\"a}ge zu erhalten und eines gl{\"u}cklichen Erfolgs in seinen Gesch{\"a}ften gewi{\ss} zu sein - Von einem alten Commis-Voyageur'' (from German: ''The Travelling Salesman - how he should be and what he has to do, in order to obtain commissions and to be assured of great success in his business - by an old Commis-Voyageur'').
The book contains a definition and an elaborate description of the problem, as well as the first ever recorded reference to it as ''the Travelling Salesman Problem'', but does not give a mathematical treatment for it. Nevertheless, the author clearly recognizes its importance, writing the following passage: 

\begin{displayquote} 
	''Business leads the traveling salesman here and there, and there is not a good tour for all occurring cases; but through an expedient choice and division of the tour so much time can be won that we feel compelled to give guidelines about this. Everyone should use as much of the advice as he thinks useful for his application. We believe we can ensure as much that it will not be possible to plan the tours through Germany in consideration of the distances and the traveling back and forth, which deserves the traveler’s special attention, with more economy. The main thing to remember is always to visit as many localities as possible without having to touch them twice.'' \cite{tspbook}
\end{displayquote}

Indeed, the main objective of the TSP is to economize, be that minimizing time, distance, fuel costs or any other expenses a travelling salesman may encounter. However, nowadays the applications of the TSP are recognized to reach far beyond travelling salesmen business. Naturally, the TSP arises in planning, transportation and logistics, problems like designing delivery routes or bus schedules are classic examples of that. There are also numerous applications in everyday life, for example for tourists wanting to visit many historical sites in a limited time, or a person running errands around town. Some other more curious and non-intuitive areas where the TSP has found applications are genetics, manufacturing, telecommunications, and neuroscience \cite{tspbook} \cite{ORlecture}.
Seeing as the TSP is so widely applicable in many areas of both industrial and day-to-day life, an effective method of solution is highly desired, yet for centuries many mathematicians have struggled to find one.

A solution to any Traveling Salesman Problem can without a doubt be obtained by simply listing all the possible tours and selecting the best one. This would indeed always work, as every scenario with Euclidian distances will always have one or more minimal cost tours. However, it must be noted, that the number of possible tours in a TSP is: \[\frac{(n-1)!}{2}.\] \noindent In this equation $n$ is the number of vertices (referred to in this work as cities). This comes from the fact that at every city the travelling salesman visits, he faces a choice of the remaining cities for his next destination, and the distance between every pair of cities is the same in each direction, forming what is called a “symmetric TSP”. Hence, for 5 cities one would find 12 possible tours, 181440 tours for 10 cities, 608225502004416000 tours for 20 and so on. For an asymmetric case, the results are even more dramatic, since the number of possible tours becomes $(n-1)!$.

Computing all the possible tours on a TSP becomes a huge burden, even when the problem in question is still relatively small. That is why mathematicians and computer scientists have been working on finding more efficient methods to solve the problem. Many methods have been developed, but not one of them is perfect and without drawbacks. One of the relatively successful recently developed methods is solving the problem via Genetic Algorithms.