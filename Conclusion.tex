\section{Conclusions}
\par
In the Introduction of this article, the following question was asked: How does a Genetic Algorithm perform when applied to the Travelling Salesman Problem?
This article has provided the general concept of a GA, together with a detailed explanation of the construction and characteristics of such an algorithm for the TSP. Furthermore, the performance of the constructed GA has been discussed on small and larger TSPs. 

\par
For small problems, the GA tends to converge to the global optimum in every run and therefore it can be said that it performs excellently. More traditional methods, like the Hungarian Algorithm and the BIP, are also capable of finding the optimal tour. However, when a problem of a larger scale is given, these traditional methods become highly inefficient. This is because of the time they need to find a solution and in some cases the construction of these methods is infeasible. Even though the GA also needs more time, when the TSP expands, it is still capable of finding a solution within minutes, which is a lot faster than the other mentioned methods. Another benefit of the GA is that it can be applied to any TSP without a reconstruction when given the neccesary data, consisting of the number of cities and their distances to each other. 

\par
The GA however, does also have some disadvantages. One example is premature convergence, which occurs when applying the GA to a very large TSP. Another disadvantage is that the exact settings for optimal performance are closely related to the GA itself and take a considerable amount of time to determine. However, this disadvantage does have a definite solution, whereas in the case of premature convergence such a solution is hard to find. It is debatable whether premature convergence is even solvable or not, some believe that solving premature convergence would also result in solving the P versus NP problem to which the TSP is famously related. 

